\documentclass[dvipdfmx,uplatex]{jsarticle}
\usepackage[utf8]{inputenc}
\usepackage{indentfirst}


\title{卒論の表題と要旨}
\author{小酒井 一朗}
\date{\today}

\begin{document}
\maketitle
\section{表題}
日本市場における生産・費用関数及び生産要素価格の関係とM\&Aが生産関数に及ぼす影響

\section{要旨}
本稿は主に4章にて構成され、順に以下の通りである。
\begin{itemize}
    \item[1章:] 日本市場における産業・セクター別生産関数の推定
    \item[2章:] 同市場・同分類における費用関数の推定
    \item[3章:] 同市場におけるM\&Aと各生産要素やTFPとの因果推論
    \item[4章:] 同市場における生産要素としての労働・資本の時系列関係性
\end{itemize}
1章・2章における生産・費用関数推定では大きく2点独自の焦点を置く。まず1点目については、通常の単なる資本・労働ではなく、それらをさらに細分化することである。続いて2点目については、これら2関数を独立した形態にて推定することである。しかし、用いる生産要素は統一することとする。\\
1章に関しては、さらに詳細に研究していく。まず、生産関数については、最初にCES、Cob-Douglas、Trans-log、線形、レオンチェフ関数を選定する。さらに、それに基づいて各生産要素の時系列的な絶対的・相対的増減について調べていく。そして、最後に生産要素を基にマークアップを推定する。\\
2章に関しても同様で、関数型の選定からその後に、各パラメータを推定していく。その後、各生産要素の要素価格と投入要素量の間の関係について考察していく。\\
3章・4章では1・2章に基づいて日本市場の生産について一層深く考察していく運びである。


\end{document}