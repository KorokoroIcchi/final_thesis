\documentclass[aspectratio=169, dvipdfmx, 11pt]{beamer}
\usepackage[utf8]{inputenc}
\usepackage{color}
\usepackage{graphicx}
\usetheme{CambridgeUS}
\usecolortheme{sidebartab}
\setbeamertemplate{navigation symbols}{}
\addtobeamertemplate{navigation symbols}{}{%
    \usebeamerfont{footline}%
    \usebeamercolor[fg]{footline}%
    \hspace{1em}%
}
\usepackage{amsmath, amssymb}
\usepackage{url}
\usepackage{ulem}
\usepackage{geometry}
\usepackage{eclclass}
\usepackage{fancybox}
\usepackage{bm}
\usepackage{array}
\setbeamercolor{button}{bg=white,fg=cyan}
\setbeamertemplate{footline}[frame number]
\geometry{
    left=7.5mm,
    right=7.5mm,
}

\title[Final Paper Plan]{Final Paper plan in Nov, 2022}
\author{Ichiro Kozakai}
\date{\today}

\begin{document}
\begin{frame}[noframenumbering, plain]
    \titlepage
\end{frame}

\begin{frame}[noframenumbering, plain]
    \tableofcontents
\end{frame}


\section{Study Question \& Motivation}
\begin{frame}
    \begin{description}
        \setlength{\itemsep}{0.3in}
        \item[My Study Question]\mbox{}\\
        \begin{itemize}
            \setlength{\itemsep}{0.1in}
            \setlength{\itemindent}{-15mm}
            \item[$\blacktriangleright$] Can labor and non-labor capital co-exist in economic activity?
            \item[$\blacktriangleright$] What M\&A brings to firms' and industries' production?
        \end{itemize}
        \item[Study Background] \mbox{}\\
            \begin{itemize}
                \setlength{\itemsep}{0.1in}
                \setlength{\itemindent}{-15mm}
                \item[$\blacktriangleright$] aaa
            \end{itemize}
    \end{description}
\end{frame}

\section[Production Function]{Estimation of Production Function with Variable Parameters}
\subsection{Select the Function}
\begin{frame}
    In this phase, I choose a type of production function, considering the function characteristics and the relation to the next section and beyond.
    \begin{itemize}
        \setlength{\itemsep}{7mm}
        \item CES : $\bm{Y}=\bm{F}\cdot\left[\sum\limits_{i}^n {a_i}{{\bm{X_{i}}}^{r}}\right]^{\frac{1}{r}}$
              \begin{itemize}
                  \setlength{\itemsep}{.1in}
                  \item[$\bigcirc$] {\color{blue} Easy to determine the substitute/complementary relation among various factors and appropriate production function}
                  \item[$\bigtriangleup$] {\color{red} Difficult to use for time-series changes because the elasticity of substitution between factors is assumed to be constant}
              \end{itemize}
        \item Cob-Douglas : $\bm{Y}=\gamma\prod\limits_{i}^{n} \bm{X_{i}}^{\alpha_{i}}$
              \begin{itemize}
                  \setlength{\itemsep}{.1in}
                  \item[$\bigcirc$] {\color{blue} Tractable}
                  \item[$\bigtriangleup$] {\color{red} Difficult to use for time-series changes because the elasticity of substitution between elements is strongly assumed to always be 1}
              \end{itemize}
    \end{itemize}
\end{frame}

\begin{frame}
    \begin{itemize}
        \setlength{\itemsep}{3mm}
        \item Trans-Log : $\ln{\bm{Y}}={\alpha_{0}}+\sum\limits_{i}^{n} \alpha_{i}(\ln{\bm{X_{i}}})+{\frac{1}{2}}\sum\limits_{i}^{n}\sum\limits_{j}^{n} {\beta_{ij}}(\ln{\bm{X_{i}}})(\ln{\bm{X_{j}}})$
              \begin{itemize}
                  \setlength{\itemsep}{.1in}
                  \item[$\bigcirc$] {\color{blue} Suitable for measuring time-series changes because it makes no a priori assumptions about the elasticity of substitution of elements}
                  \item[$\bigtriangleup$] {\color{red} Awkward}
              \end{itemize}
        \item Linear : $\bm{Y}=\sum\limits_{i}^{n} {\alpha_{i}}{\bm{X_{i}}}$
              \vspace{-4mm}
        \item Leontief : $\bm{Y}=min\{{\frac{\bm{X_{1}}}{\alpha_{1}}}, \cdots, {\frac{\bm{X_{n}}}{\alpha_{n}}}\}$
              \begin{itemize}
                  \setlength{\itemsep}{.1in}
                  \vspace{2mm}
                  \item[$\bigcirc$] {\color{blue} Tractable}
                  \item[$\bigtriangleup$] {\color{red} Very restrictive and limited number of cases}
              \end{itemize}
    \end{itemize}
\end{frame}

\begin{frame}
    \vspace{4mm}
    \textbf{\Large{Reference}}\\
    \vspace{8mm}
    \hyperlink{file:///C:/Users/ichir/Downloads/keizai281-14.pdf}{\beamergotobutton{\dotuline{生産関数推定について:手法に関する考察と規制緩和への示唆(中村豪)}}}\\
    \vspace{3mm}
    \hyperlink{https://www.maff.go.jp/primaff/kanko/nosoken/attach/pdf/198601_nsk40_1_01.pdf}{\beamergotobutton{\dotuline{2段階CES型生産関数の計測と誘発的技術変化仮説の検証(川越俊彦)}}}\\
    \vspace{3mm}
    \hyperlink{https://core.ac.uk/download/pdf/286932843.pdf}{\beamergotobutton{\dotuline{フレキシブル関数の理論(浜口登)}}}\\
    \vspace{3mm}
    \hyperlink{https://www1.doshisha.ac.jp/~kmiyazaw/undergraduate/Elasticity_of_substitution.pdf}{\beamergotobutton{\dotuline{要素代替の弾力性(宮澤和俊)}}}\\
    \vspace{3mm}
    \hyperlink{https://www.jstage.jst.go.jp/article/srs1970/25/1/25_1_147/_pdf/-char/ja}{\beamergotobutton{\dotuline{トランス・ログ型関数による航空輸送産業の費用構造の分析(衣笠達夫)}}}\\
    \vspace{3mm}
    \hyperlink{https://www.rieti.go.jp/jp/publications/dp/17e083.pdf}{\beamergotobutton{\dotuline{Decomposition of Aggregate Productivity Growth with Unobserved Heterogeneity(笠原博幸、西田充邦、鈴木通雄)}}}
\end{frame}

\subsection{Decomposition of Aggregate Factor}
\begin{frame}
    \begin{columns}
        \begin{column}{0.54\textwidth}
            \begin{center}
                \begin{classify}{\textbf{\Large \color{blue}{Labor Capital}}}
                    \class{
                        \begin{classify}{Education}
                            \class{{\color{red}High-Educated/Skilled}}
                            \class{{\color{red}Low-Educated/Unskilled}}
                        \end{classify}
                    }
                    \class{
                        \begin{classify}{Employment}
                            \class{
                                \begin{classify}{Contract Type}
                                    \class{{\color{red}Permanent}}
                                    \class{{\color{red}Contract}}
                                    \class{{\color{red}Part-Time}}
                                \end{classify}
                            }
                            \class{
                                \begin{classify}{Career}
                                    \class{{\color{red}Proper}}
                                    \class{{\color{red}Job-Changer}}
                                    \class{{\color{red}Loaned Staff}}
                                \end{classify}
                            }
                        \end{classify}
                    }
                \end{classify}
            \end{center}
        \end{column}
        \begin{column}{0.02\textwidth}
            \rule{.1mm}{.9\textheight}
        \end{column}
        \begin{column}{0.44\textwidth}
            \begin{center}
                \begin{classify}{\textbf{\Large \color{blue}Non-Labor Capital}}
                    \class{
                        \begin{classify}{Liquidity}
                            \class{{\color{red}Fixed Assets}}
                            \class{{\color{red}Current Assets}}
                            \class{{\color{red}Deferred Assets}}
                        \end{classify}
                    }
                    \class{
                        \begin{classify}{Tangibility}
                            \class{{\color{red}Tangible Assets}}
                            \class{{\color{red}Intangible Assets}}
                        \end{classify}
                    }
                \end{classify}
            \end{center}
        \end{column}
    \end{columns}
\end{frame}

\subsection{Industry/Sector Classification}
\begin{frame}
    I wanna classify firms/Corporation into some Industry or Sector with easy and versatile ways. Some classification-criteria candidates are as following;\\
    \vspace{8mm}
    \hyperlink{https://www.spglobal.com/spdji/jp/landing/topic/gics/}{\beamergotobutton{\dotuline{\Large{S\&P Dow Jones Indexes(GICS)}}}}\\
    \begin{itemize}
        \item[$\bullet$] \hyperlink{https://www.jpx.co.jp/markets/indices/carbon-efficient/nlsgeu000003b0r8-att/sp-jpx-carbon-efficient-index-constituent-report.pdf}{\beamerbutton{GICS for Japan Market ver. 2022}}
    \end{itemize}
    \vspace{6mm}
    \hyperlink{https://www.nikkei.com/telecom/industry_l}{\beamergotobutton{\dotuline{\Large{日経業種分類}}}}\\
    \vspace{6mm}
    \hyperlink{https://www.soumu.go.jp/toukei_toukatsu/index/seido/sangyo/02toukatsu01_03000044.html}{\beamergotobutton{\dotuline{\Large{日本標準産業分類}}}}\\
    \vspace{6mm}
    \hyperlink{https://management-accounting.biz/industry-type-2/}{\beamergotobutton{\dotuline{\large{Other ways \& reference}}}}
\end{frame}

\subsection{Time-Series Change and Factor Scale}
\begin{frame}
    In this phase, I organize the time-series changes of parameters of production function and the time-series trends of abstract/relevant amounts for each production factors.
\end{frame}

\subsection{Estimates of markup rate}
\begin{frame}
    Referring to \hyperlink{https://repository.tku.ac.jp/dspace/bitstream/11150/11249/1/keizai299-11.pdf}{\beamerbutton{\dotuline{生産関数を用いたマークアップ率の計測に関する検証(中村豪)}}}, I measures the markup rate and its relation with production parameters for each industries/sectors.
\end{frame}

\section[Cost Function]{Estimation of Cost Function Independently}
\subsection{Select the Function}
\begin{frame}
    In this section, I estimate the Cost Function independently of production function with same factors.\\
    The reference is as following;\\
    \vspace{10mm}
    \hyperlink{https://doshisha.repo.nii.ac.jp/?action=repository_action_common_download&item_id=22965&item_no=1&attribute_id=28&file_no=1}{\beamergotobutton{国立大学の費用関数:トランスログ・コストシェアモデルによる同時推定(北坂真一)}}\\
    \vspace{6mm}
    \hyperlink{https://core.ac.uk/download/pdf/71796667.pdf}{\beamergotobutton{実証分析における生産,費用関数(中西泰夫)}}
\end{frame}

\subsection{Estimation the factors price}
\begin{frame}
    In this phase, I will estimate the factor price with cost function estimate above or IO-table if possible.
\end{frame}

\section[Easy Causal Inference]{Causal Inference between M\&A and Factor Productivity}
\begin{frame}
    \begin{itemize}
        \setlength{\itemsep}{8mm}
        \item In the former paper written with seminar members, we investigated why M\&A.
        \item In this paper, I study what M\&A brought to firms' production/productivity with statistic causal inference.
        \item At last, I will study the connection between M\&A and firms'/Sector or industry's markup.
    \end{itemize}
\end{frame}

\section[Capital or/and Labor]{Consideration of Time-Series Relationship between Capital and Labor}
\begin{frame}Based on the aforementioned study, I put my own answers to my study questions.
    \vspace{12mm}
    \begin{itemize}
        \setlength{\itemsep}{8mm}
        \item Human labor and non-labor capital are substitute? or complementary?
        \item How No. of entries and exit of companies and decrease No. of companies in a market affects their productivity or labor-capital relation?
    \end{itemize}

\end{frame}
\section{Data}
\begin{frame}
    \begin{itemize}
        \setlength{\itemsep}{5mm}
        \item Bloomberg
        \item DBJ
        \item Nikkei needs
        \item IO table
    \end{itemize}
\end{frame}
\end{document}